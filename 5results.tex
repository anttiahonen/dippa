\chapter{Results}
\label{chapter:results}

This chapter covers the results and analysis of the case study. First the results of interview for demographic purposes are
presented. Second, survey results regarding JUnit compared to Spock and Spectrum are analyzed. Third, the participant interview
about BDD testing framework benefits and drawbacks are discussed. After this, the test code analysis is presented and finally
with all the above mentioned results, Spock and Spectrum are compared against each other.


%-Define the population to which inferential statistics and predictive models apply.\newline
%-RQ1 and RQ2 answered by comparing initial and post surveys\newline
%-Comparing Spock and Spectrum vs each other\newline
%-RQ3 answered by comparing observed actual test implementation changes\newline

\section{First interview: Demographics and projects}
\label{section:demographics}

\section{Surveys}
\clearpage
\newgeometry{left=1cm, right=1cm}
    \begin{table}
        \resizebox{\textwidth}{!}{%
            \begin{tabular}{p{20.0cm}*{7}{p{2.0cm}}}
            \hline
            \textbf{Question} & \textbf{Answer options} &  &  &  &   &  & \\ \hline
            \textbf{Q1: How do you spend your software development time (in percentages)} & Participant A & Participant B & Participant C & Average & & \\
            & & & & & & & \\
            1. Writing new code & 20\% & 40\% & 15\% & 25\%  \\
            2. Writing new tests & 20\% & 25\% & 20\% & 21.67\% \\
            3. Debugging/fixing & 30\% & 25\% & 25\% & 26.67\% \\
            4. Refactoring & 20\%  & 10\% & 30\% & 23.33\% \\
            5. Other & 10\% & 0\% & 10\% & 6.67\% \\
            & \\ \hline

            \textbf{Q2: How do you spend your low level automated testing time} & Participant A & Participant B & Participant C & Average & & \
            & & & & & & & \\
            1. How much approximately you use time per test case (minutes)? & 30 min & 30 min & 30 min & 30 min \\
            2. How much of your initial effort goes to thinking about test case content without implementation (percentage)? & 50\% & 20\% & 15\% & 28.33\% \\
            3. How much of your initial effort goes to initial test case structuring and implementation (percentage)? & 50\% & 80\% & 85\% & 71.67\% \\
            4. How much of your overall testing effort goes to refactoring test code (percentage)? & 50\% & 5\% & 30\% & 28.33\% \\
            & \\ \hline

            \textbf{Q3: How important are the following aspects for you when you write new low level tests?} & & & & & & \\
            & Not at all & Low \newline importance & Slightly important & Neutral & Moderately important & Very \newline important & Extremely important \\
            1. Code coverage & 0 & 0 & 0 & ABC & 0 & 0 & 0 \\
            2. Capturing all behavior of unit/feature with tests or assertions & 0 & 0 & B & A & 0 & C & 0 \\
            3. Execution speed & 0 & C & 0 & 0 & B & A & 0 \\
            4. Robustness against code changes (i.e., test does not break easily) & 0 & 0 & 0 & 0 & ABC & 0 & 0 \\
            5. How realistic the test scenario is & 0 & 0 & AB & 0 & 0 & C & 0 \\
            6. How easily faults can be localised/debugged if the test fails & 0 & 0 & B & 0 & AC & 0 & 0 \\
            7. How easily the test can be updated when the underlying code changes & 0 & C & A & 0 & B & 0 & 0 \\
            8. Sensitivity against code changes (i.e., test should detect even small code changes) & 0 & AB & 0 & C & 0 & 0 & 0 \\
            & \\ \hline

            & & & & & & \\
            & Very easy & Easy & Moderate & Hard & Very hard & & \\
            \textbf{Q4: How difficult is it for you to understand a low level test?} & 0 & 0 & BC & A & 0 \\
            & \\ \hline

            \textbf{Q5: In low level testing, how difficult is it for you to} & & & & & & \\
            & Very easy & Easy & Slightly easy & Moderate & Slightly hard & Hard & Very hard \\
            1. Structure and write information to context of test? & 0 & A & 0 & 0 & BC & 0 & 0 \\
            2. Structure and write information to stimulus of test? & 0 & 0 & 0 & ABC & 0 & 0 & 0 \\
            3. Structure and write information to assertions of test? & 0 & B & C & 0 & A & 0 & 0 \\
            4. Read test case structure for information about context of test? & 0 & 0 & C & 0 & B & A & 0 \\
            5. Read test case structure for information about stimulus of test? & 0 & 0 & C & B & A & 0 & 0 \\
            6. Read test case structure for information about assertions of test? & 0 & B & C & 0 & 0 & A & 0 \\
            & \\ \hline

            & & & & & & \\
            & Not at all & Hardly informative & Slightly informative & Somewhat informative & Moderately informative & Very informative & Extremely informative \\
            \textbf{Q6: How informative you usually find the test case output?} & 0 & A & 0 & C & B & A & 0 \\
            & \\ \hline

            \textbf{Q7: How much are the following repetition reducing techniques used in your low level testing?} & & & & & & \\
            & Never & Very rarely & Rarely & Occasionally & Frequently & Very \newline frequently & Always \\
            1. Extract method (custom helper methods) & 0 & 0 & 0 & 0 & AB & C & 0 \\
            2. Lifecycle hooks Before/After -class & 0 & 0 & 0 & C & AB & 0 & 0 \\
            3. Lifecycle hooks Before/After (each) & 0 & 0 & 0 & 0 & AB & C & 0 \\
            4. Automatic test case generation via test case parametrization & 0 & A & B & C & 0 & 0 & 0 \\
            5. Common test initializer class inheritance & 0 & AC & B & 0 & 0 & 0 & 0 \\
            & \\ \hline

            & & & & & & \\
            & Never & Rarely & Sometimes & Fairly often & Always & & \\
            \textbf{Q8: How often do you add/write documentation comments to low level test cases?} & 0 & C & 0 & AB & 0 \\
            & \\ \hline

            & & & & & & \\
            & Never & Rarely & Sometimes & Fairly often & Always & & \\
            \textbf{Q9: When you make changes to low level tests, how often do you comment the changes (or update existing comments)?} & 0 & C & 0 & AB & 0 \\
            & \\ \hline

            \textbf{Q10: In unit testing, how many} & & & & & & \\
            & 0 & 1 & 2-3 & 4-5 & 6-7 &  8-9 & 10 or more \\
            1. Test methods do you usually write per class method? & 0 & 0 & AB & C & 0 & 0 & 0 \\
            2. Assertions do you usually write per test method? & 0 & 0 & AB & C & 0 & 0 & 0 \\
            & \\ \hline

            & & & & & & \\
            & Mockito & jMock & Powermock & Easymock & Other & & \\
            \textbf{Q11: In unit testing, what mocking library do you normally use?} & ABC & 0 & 0 & 0 & 0 \\
            & \\ \hline

            \textbf{Q12: In unit testing, how difficult you find it to} & & & & & & \\
            & Very easy & Easy & Slightly easy & Moderate & Slightly hard & Hard & Very hard \\
            1. Mock objects? & 0 & BC & A & 0 & 0 & 0 & 0 \\
            2. Stub method calls? & 0 & 0 & AC & 0 & B & 0 & 0 \\
            3. Verify mock object actions? & 0 & BC & 0 & A & 0 & 0 & 0 \\
            & \\ \hline

            & & & & & & \\
            & Before implementation & During implementation & After implementation & \\
            \textbf{Q13: When do you add automated unit tests for developed code?} & 0 & BC & A \\
            & \\ \hline

            \end{tabular}}
            \caption {JUnit developer low level testing practice questions with response data} \label{tab:junit-pt1}
    \end{table}
    \clearpage
    \begin{table}[H]
        \resizebox{\textwidth}{!}{%
            \begin{tabular}{p{20.0cm}*{7}{p{2.0cm}}}
            \hline
            \textbf{Question} & \textbf{Answer options} &  &  &  &   &  & \\ \hline

            \textbf{Q14: Please indicate your level of agreement with the following statements} & & & & & & \\
            & Strongly disagree & Disagree & Somewhat disagree & Neither agree nor disagree & Somewhat agree & Agree & Strongly agree \\
            1. Writing low level tests is difficult & 0 & 0 & C & 0 & AB & 0 & 0 \\
            2. I enjoy writing low level tests & 0 & B & 0 & A & 0 & C & 0 \\
            3. I would like to have more tool support when writing low level tests & 0 & 0 & B & A & C & 0 & 0 \\
            4. I would like to have more low level tests & 0 & 0 & 0 & ABC & 0 & 0 & 0 \\
            5. Maintaining low level tests is difficult & 0 & 0 & C & 0 & B & A & 0 \\
            6. I think my low level tests will help other developers to understand the implemented unit/feature better & 0 & 0 & B & 0 & A & C & 0 \\
            7. Low level automated testing helps me find defects in the code before other quality assurance phases & 0 & 0 & 0 & 0 & B & C & A \\
            8. JUnit promotes me to write high quality test code & 0 & 0 & AC & B & 0 & 0 & 0 \\
            & \\ \hline

            \textbf{Q15: Please indicate your level of agreement with the following statements} & & & & & & \\
            & Strongly disagree & Disagree & Neutral & Agree & Strongly agree & \\
            1. Overall, low level tests help me produce higher quality code & 0 & 0 & 0 & B & AC \\
            2. Maintaining good low level test cases and their documentations is important to the quality of a system & 0 & 0 & 0 & AB & C \\
            & \\ \hline

            \end{tabular}}
            \caption {Likert scale questions with response data about developer perception towards JUnit} \label{tab:junit-pt2}
    \end{table}
    \vspace{20px}
    \begin{table}[H]
        \resizebox{\textwidth}{!}{%
            \begin{tabular}{p{17.0cm}*{11}{p{1.55cm}}}
            \hline
            \textbf{Question} & \textbf{Answer options} &  &  &  &  &  &  &  &  & \\ \hline

            \textbf{Q16: How likely are you to} &  &  &  &  &  &  &  &  &  &  \\
            & 0 & 1 & 2 & 3 & 4 & 5 & 6 & 7 & 8 & 9 & 10 \\ \hline
            1. Recommend low level automated testing for colleague as a software development practice? & 0 & 0 & 0 & 0 & 0 & 0 & 0 & 0 & B & 0 & AC \\
            2. Recommend testing framework JUnit for future Spring projects where you take part in existing project? & 0 & 0 & 0 & 0 & 0 & 0 & 0 & C & 0 & B & A \\
            3. Take testing framework JUnit in use for future Spring projects where you have technical lead role in a new starting project? & 0 & 0 & 0 & 0 & 0 & 0 & 0 & 0 & C & B & A \\
            & \\ \hline

            \end{tabular}}
            \caption {NPS questions with response data about developer loyalty towards low level automated testing with JUnit } \label{tab:junit-pt3}

    \end{table}
    \clearpage
\restoregeometry

\section{Second interview: BDD framework feedback}

\section{Test code analysis}

\section{Comparison of BDD testing frameworks}

