\chapter{Conclusions}
\label{chapter:conclusions}
First in this chapter, the summary of case study and its results are presented. Second, the results are presented
from the viewpoint of comparing studied implementation level BDD-testing frameworks. Third, possible directions for future work
on the studied topic are discussed.

\section{Summary}
With today's trends in software development, agile practices are used more and more in producing software. What these practices
usually have in common, is the incrementally built quality where automated testing done by developers plays a crucial role.
Traditionally developers use \textit{xUnit family} testing frameworks for low-level automated testing. Although easy
to get started, research has shown multiple possible problematic areas in the practice of unit testing with the said
frameworks. This thesis studied alternative ways for low-level testing in the JVM context for testing Java-code.
This was conducted as a case study in multi-project industry context.
Idea was to see whether implementation level BDD-testing frameworks could help in problematic testing areas and how they
change the practices and perception of low-level testing Java-code in general.

\textbf{Practice changes} in low-level testing with BDD-testing frameworks included developers writing \textit{more granular}
and \textit{self-documenting} test cases. Especially the analyzed test code changes provided strong evidence for these changes.
Other changes included developers feeling that it was \textit{easier to identify separate test parts} than before with
\textit{JUnit}. There was also evidence to support \textit{easier understanding of tests, more informative test output} and
possibly \textit{easier ways to remove repetition} from test code. Practice changes with BDD-testing frameworks were not only
positive, as studied framework \textit{Spectrum} displayed a somewhat \textit{steep learning curve} with more time
and effort needed in low-level testing compared to \textit{JUnit}.

\textbf{Perception changes} in low-level testing with BDD-testing frameworks displayed some \textit{rise in enjoyment} in testing.
There was also change in perception where two out of three practitioners perceived test done with BDD-testing frameworks
more \textit{understandable} and \textit{maintanable} than before. Amongst the same participants, this was also accompanied
together with the perception that BDD-testing frameworks \textit{promote to write higher quality} test code than \textit{JUnit}.
One of the participants perceived changes in testing as neutral or slightly negative during the study period.

In conclusion, these results are quite promising as possible solutions to common problems found in unit testing, such as \textit{low enjoyment} in testing
together with \textit{hard readability} and \textit{maintainability} of tests. Future work section discusses in more detail
how the use of implementation level BDD-testing frameworks in low-level testing could be studied further. Before that,
next section summarizes the test changes as a comparison between used BDD-testing frameworks.

\section{Comparison of BDD-testing frameworks}
{\renewcommand{\arraystretch}{1.3}
\begin{table}[H]
    \resizebox{\textwidth}{!}{%
        \begin{tabular}{p{9.0cm}*{2}{p{3cm}}}

        \headcol \textbf{Studied aspect} & \textbf{Spectrum} & \textbf{Spock} \\ \hline

        \rowcol Learning curve & Slow & Fast \\
        More granular test cases & Yes & Yes \\
        \rowcol Easier to structure tests & Potentially & Yes \\
        Easier to understand tests & Potentially & Yes \\
        \rowcol More enjoyable to test code & Potentially & Yes \\
        More informative test output & Yes & Yes \\
        \rowcol More maintainable tests & Potentially & Yes \\
        More self-documenting tests & Yes & Yes \\
        \rowcol Framework and tool support & Adequate & Good \\
        \end{tabular}}
        \caption {Summary of studied aspects of low-level testing with BDD frameworks } \label{tab:bdd-comparison}
\end{table}
}

Table \ref{tab:bdd-comparison} summarizes the most important findings in this study from the viewpoint of the
BDD-testing frameworks. There are two keypoints when comparing the alternatives: \textit{Spock} offers more mature
and easier to begin starting point for \textit{Spring Framework} low-level testing than \textit{Spectrum}.

Both produce more
granular test cases than \textit{JUnit}, where \textit{Spock} achieves this with DDT and \textit{Given-When-Then} blocks. \textit{Spectrum} does this with more individually separated
code examples describing behavior. \textit{Spock} and its DDT should promote more easily maintainable tests, whereas \textit{Spectrum}
way of writing tests promotes individually passing or failing test conditions. \textit{Spectrum} and
its nested example groups with lifecycle hooks can be used for removing repetition from the code.

There is evidence to support that both can potentially allow more easily to structure and understand tests than \textit{JUnit}. Although, the structure of
\textit{Spock} doesn't contain nesting and as such might allow more easily to understand individual tests.
With \textit{Spectrum}, there exists a longer learning curve to 1earn how to structure and understand the nested structure
properly. Especially the Java lambda-features used to structure \textit{Spectrum} tests were found troublesome within the two months
study time.
Both BDD frameworks
promote to write more self-documenting test cases than before. This is especially visible with \textit{Spectrum}, where it
can reduce the need for explicit commenting with test info embedded into test descriptive naming. This self-documenting
aspect of tests is also visible in more information displayed in test output.

Summed up, these changes make both BDD frameworks good candidates to rise enjoyment in low-level testing compared to traditional
\textit{JUnit} testing. Although with \textit{Spectrum}, the rise in enjoyment was not perceived unanimously. \textit{Spock} could potentially rise enjoyment in low-level
testing more quickly than \textit{Spectrum}.
\textit{Spock}'s structure seems easier to learn when switching from \textit{JUnit} and thus resulting more quickly in well structured tests.


\section{Future work}
Due to the limited scope of this study, together with constraints of master thesis, this case study has acted as preliminary
research on the topic of comparing traditional unit testing framework \textit{JUnit} against
implementation level BDD-testing frameworks. This was conducted in an environment where BDD-testing frameworks were used
as an alternative for \textit{JUnit} in automated unit and integration testing, without the practice of BDD.
There exists interesting possible research to continue
on the topic of studying the use of implementation level BDD-testing frameworks for low-level testing.

Potential future work on the topic could include replicating this study on a larger scale on the JVM enviroment.
This kind of
study could also possibly be conducted in different environment, such as the \textit{.NET} environment. With
larger projects and number of study participants, it would be interesting to see whether the larger scale results
follow the same trend as results gathered in this study.
As many of the \textit{xSpec family} drawbacks in this study was found out to
come from the Java language features used in \textit{Spectrum}, it could be one possible change in future studies to use
alternative \textit{xSpec family} framework.
For example framework from a dynamic programming language could be used, such as earlier demonstrated \textit{RSpec} from JRuby.

Another future work possibility could be to further explore the studied aspects of this research in multiple isolated settings.
This means using the same kind of survey that was used in this thesis to study for example long time Java developers using \textit{JUnit},
Ruby developers using \textit{RSpec} and Groovy developers using \textit{Spock}.
This kind of research could be done with a large number of participants and the results
could be used to compare \textit{xUnit family}, \textit{xSpec family} and \textit{Gherkin family} for low-level testing on
a grand scale.
It would offer insight on low-level testing
practices and perception towards it without the possible resistance to change that can surface in replicating exact
same study as in this thesis. The test code analysis could also be done on a grand scale, for example mining
public \textit{GitHub} repositories and their \textit{JUnit, RSpec} or \textit{Spock} tests to calculate different metrics.
