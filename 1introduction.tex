\chapter{Introduction}
\label{chapter:intro}

This is my master's thesis, and I am very proud of it.  Of course,
when I write my \emph{real} master's thesis, I will not use the
singular pronoun \emph{I}, but rather try to avoid referring to myself
and speak of the research \emph{we} have conducted---I rarely work
alone, after all.  Yet, both \emph{I} and \emph{we} are correct, and
it depends on the instructor and the supervisor (of course from you,
too), which one they would prefer. Anyway, the tense should be active,
and passive sentenses should be avoided (especially, writing sentences
where the subject is presented with by preposition), so often you
cannot avoid choosing between the pronouns. Life is strange, but there
you have it.

By the way, the preferred order of writing your master's thesis is
about the same as the outline of the thesis: you first discover your
problem and write about that, then you find out what methods you
should use and write about that.  Then you do your implementation, and
document that, and so on.  However, the abstract and introduction are
often easiest to write last.  This is because these really cover the
entire thesis, and there is no way you could know what to put in your
abstract before you have actually done your implementation and
evaluation. Rarely anyone write the thesis from the beginning to the
end just one time, but the writing is more like process, where every
piece of text is written at least twice. Be also prepared to delete
your own text. In the first phase, you can hide it into comments that
are started with \% but during the writing, the many comments should be
visible for your helpers, the instructor and supervisor.

The introduction in itself is rarely very long; two to five pages often
suffice.


\section{Problem statement}

Undergraduate students studying technical subjects do not consider typography
very interesting these days, and therefore the typographical quality of many
theses is unacceptably low. 
We plan to rectify this situation somewhat by providing a decent-quality
example thesis outline for students.
We expect that the typographical quality of the master's theses will
dramatically increase as the new thesis outline is taken into use.

\section{Helpful hints}

Read the information from the university master's thesis
pages~\cite{ThesisInstructions} before starting the thesis.  You
should also go through the thesis grading
instructions~\cite{ThesisGrading} together with your instructor and/or
supervisor in the beginning of your work.

\section{Structure of the Thesis}
\label{section:structure} 

You should use transition in your text, meaning that you should help
the reader follow the thesis outline. Here, you tell what will be in
each chapter of your thesis.


-General thoughts about the thesis
-Problem statement
-Research questions
-Structure of the thesis

