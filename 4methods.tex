\chapter{Methods}
\label{chapter:methods}
This chapter discusses the design of the study. First, research questions are introduced.
Second, research hypothesis is made based on the reviewed practices and related research findings introduced in the chapter \ref{chapter:background}.
Third, the empirical study and its methodology are explained in detail. Finally, limitations to the study methods are discussed.

\section{Research questions}
This thesis studies low level testing done with testing framework JUnit and how it changes after putting
an implementation level BDD testing framework into operation. The scope in research is testing Java-code. The following
research questions are aimed to highlight the change in low level testing after the testing framework change:
\begin{enumerate}
\item \textbf{RQ1}: How does behavior driven testing frameworks change developer practices working with automated low level testing compared to xUnit (JUnit) framework?
\item \textbf{RQ2}: How does behavior driven testing frameworks change developer perception of working with automated low level testing compared to xUnit (JUnit) framework?
\item \textbf{RQ3}: How does behavior driven testing frameworks change written low level test cases and test code coverage compared to xUnit (JUnit) framework?
\end{enumerate}

\section{Research hypothesis} %1p
    Discussion about research hypothesis (BD-testing vs JUnit) and how it relates to problems found (8) and benefits mentioned in section 6-7.\newline
    -Developers will write more granular test cases\newline
        %-Ubiquitous language
        %-BDD Helps to shift viewpoint from 1-1 relationship between test-code and only one test method per function
    -Developers will find easier to understand test cases\newline
        %-More comments, longer descriptions naturally in use through ubiquitous language
        %-Test Code should describe the behaviors of the object
    -Developers will find it easier to maintain code\newline
         %-Test Code should be part of system’s documentation
         %-Test Code should have less repetition
    -Developers will perceive working with low level automated testing more enjoyable\newline

\section{Emperical study}
-Define the type of study
\subsection{Process for selecting teams and developers}
-Define the process by which the subjects and objects were selected.\newline
\subsection{Interview for demographic purposes}
-Survey participant semi-structured\newline
-Interview questions shown\newline
\subsection{Questionnaire about low level automated testing}
    -Existing base questions explained with references to original publications\newline
    -NPS explained\newline
    -Implementation of JUnit survey / screenshots / questions shown\newline
    -Implementation of Spock/Spectrum survey / screenshots / questions shown\newline\newline
\subsection{Quantitative research}
-Mine repositories test code\newline
    -Define all software measures fully, including the entity, attribute, unit and counting rules.

\section{Threats to validity}
-Response bias\newline
-Presence of a "champion" in influencing the use & first impressions\newline
-Threats to validity of questionnaire\newline\newline
\section{Limitations of the case study}
-limited time\newline
-limited number of participants\newline
-limited amount of new test cases\newline


