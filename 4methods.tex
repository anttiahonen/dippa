\chapter{Methods}
\label{chapter:methods}
This chapter discusses the design of the study. First, research questions are introduced. Second, related research on low
level testing practices and perception are discussed together with available thesis studies done on BDD.
Third, research hypothesis is made based on the practices reviewed in the chapter \ref{chapter:background} and with the related research findings.
After this, the empirical study and quantitative analysis methodology are explained in detail. Finally, limitations
to the study methods are discussed.

\section{Research questions}
This thesis studies low level testing done with testing framework JUnit and how it changes after putting
an implementation level BDD testing framework into operation. The scope in research is testing Java-code. The following
research questions are aimed to highlight the change in low level testing after the testing framework change:
\begin{enumerate}
\item \textbf{RQ1}: How does behavior driven testing frameworks change developer practices working with automated low level testing compared to xUnit (JUnit) framework?
\item \textbf{RQ2}: How does behavior driven testing frameworks change developer perception of working with automated low level testing compared to xUnit (JUnit) framework?
\item \textbf{RQ3}: How does behavior driven testing frameworks change written low level test cases and test code coverage compared to xUnit (JUnit) framework?
\end{enumerate}

\section{Related research} % 2p
    Although BDD and BDD testing frameworks have been around over a decade, empirical research made on the topic is limited.
    BDD testing frameworks are in heavy use in certain programming languages and frameworks, such as \textit{RSpec} in Ruby on Rails testing~\cite{lerner2009forge},
    \textit{Jasmine} in JavaScript testing~\cite{amodeo2015learning} and \textit{Spock} in Groovy testing. There exists no exact research on how
    popular practicing BDD is on the mentioned environments, but the reality probably is that these BDD implementation level
    frameworks are used largely only for testing purposes, not practicing BDD. Therefore, the findings in studies done on low level testing
    are the most important ones regarding this thesis. The scope is to study the changes in low level testing from introducing
    BDD implementation level testing frameworks without the practice of BDD.


    A Survey on Unit Testing Practices and Problems:\newline
    -Developers are striving to find realistic scenarios.\newline
    -Isolating unit under test was found hard (mocking etc)\newline
    -There clearly is potential for unit testing research to help developers produce better tests that make debugging and fixing easier\newline
    -Understanding code is bigger problem than understanding test code\newline
    --> If good tests can be generated, then these may help in understanding the code.\newline
    -Only half of the respondents enjoy writing unit tests\newline
    --> There is a need for tools that rise the enjoyment\newline
    -Maintaining unit tests is found harder than writing unit tests\newline
    --> Need for easier maintaining. Can better readability, Easy parametrization and code repetition removal practices help?\newline\newline

    Automatically Documenting Unit Test Cases:\newline
    -Majority of developers (60.38percent) found understanding of unit test cases to be at least moderately difficult\newline
    -Developers find up-to-date documentation and comments within test cases to be useful\newline
    -Writing comments to unit tests is a practice rarely or never done\newline
    -In order to effectively maintain test cases, it is important that developers understand the impact of each unit
     test case and the particular functionality that it aims to test.\newline
    -Developers feel that tests should be self-documentating\newline
    -More than half of the developers indicated a difficulty of moderate to very hard in terms of understanding unit tests.\newline
    -Emphasizing this importance, we observed that 89.15percent of developers agree or strongly agree that maintaining test cases
      impacts the quality of the system. This suggests that developers could benefit from tools that support them in maintaining unit test
       cases during software evolution and maintenance\newline\newline

   Tsiigaa myos Berner ja testware maintenance problems~\cite{berner2005observations}

   Per Runeson: A Survey of Unit Testing Practices\newline
   -Unit tests are documented in test code rather than in text\newline
   -Unit test motivation in agile: test suites could function as a specification\newline
   -Maintaining was found to take much effort\newline
   -Developer motivation working with unit tests needs improvement\newline

   -Analysis of problems found from automated testing surveys\newline
   -Reference to future research introduced in thesis work\newline

   Tänne dippatyö-analyysit

   Dippatyöstä TOWARDS AN EMPIRICAL EVALUATION
               OF BEHAVIOUR - DRIVEN DEVELOPMENT\newline
   -Suora lainaus: The benefits of developing a system written in a static language such as Java, while specifying its
    behaviours using a more flexible dynamic language such as JRuby could be analyzed.\newline\newline
\section{Research hypothesis} %1p
    Discussion about research hypothesis (BD-testing vs JUnit) and how it relates to problems found (8) and benefits mentioned in section 6-7.\newline
    -Developers will write more granular test cases\newline
        %-Ubiquitous language
        %-BDD Helps to shift viewpoint from 1-1 relationship between test-code and only one test method per function
    -Developers will find easier to understand test cases\newline
        %-More comments, longer descriptions naturally in use through ubiquitous language
        %-Test Code should describe the behaviors of the object
    -Developers will find it easier to maintain code\newline
         %-Test Code should be part of system’s documentation
         %-Test Code should have less repetition
    -Developers will perceive working with low level automated testing more enjoyable\newline

\section{Emperical study}
-Define the type of study
\subsection{Process for selecting teams and developers}
-Define the process by which the subjects and objects were selected.\newline
\subsection{Interview for demographic purposes}
-Survey participant semi-structured\newline
-Interview questions shown\newline
\subsection{Questionnaire about low level automated testing}
    -Existing base questions explained with references to original publications\newline
    -NPS explained\newline
    -Implementation of JUnit survey / screenshots / questions shown\newline
    -Implementation of Spock/Spectrum survey / screenshots / questions shown\newline\newline



\section{Quantitative research}
-Mine repositories test code\newline
    -Define all software measures fully, including the entity, attribute, unit and counting rules.

\section{Threats to validity}
-Response bias\newline
-Presence of a "champion" in influencing the use & first impressions\newline
-Threats to validity of questionnaire\newline\newline
\section{Limitations of the case study}
-limited time\newline
-limited number of participants\newline
-limited amount of new test cases\newline


