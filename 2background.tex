\chapter{Background}
\label{chapter:background} 

\section{JVM}
\section{Quality assurance}
\section{Automated testing}
    \subsection{Levels of automation}
    \subsection{Low level testing explained}
    \subsection{Benefits and drawbacks}
    %Microsoft\newline
    %-20.9percent decrease in test defects\newline
    %-Cost approximately 30percent more development time\newline
    %-Relative decrease in defects found by customer in first 2 years of use\newline
    %-Developers feel more confident in refactoring and modifying code due to precense of automated unit tests\newline
\section{xUnit testing frameworks}
-JUnit\newline
-Java Spring testing with with JUnit\newline
\section{Test Driven Development}
    \subsection{Definition}
    \subsection{Benefits and drawbacks}
\section{Behavior Driven Development}
    \subsection{Definition}
    \subsection{BDD vs TDD}
    \subsection{Levels of specification}
    \subsection{Tools}
    \subsection{Benefits and drawbacks}
\section{Behavior Driven low level testing frameworks}
    \subsection{Spec family}
            -Rspec and Spectrum\newline
            -Rspec / BDD Helps to shift viewpoint from 1-1 relationship between test-code and only one test method per function
    \subsection{Spock}
\section{Related research}
    -Analysis of problems found from automated testing surveys\newline
    -Reference to future research introduced in thesis work\newline

    Dippatyöstä TOWARDS AN EMPIRICAL EVALUATION
                OF BEHAVIOUR - DRIVEN DEVELOPMENT\newline
    -Suora lainaus: The benefits of developing a system written in a static language such as Java, while specifying its
     behaviours using a more flexible dynamic language such as JRuby could be analyzed.\newline\newline

    A Survey on Unit Testing Practices and Problems:\newline
    -Developers are striving to find realistic scenarios.\newline
    -Isolating unit under test was found hard (mocking etc)\newline
    -There clearly is potential for unit testing research to help developers produce better tests that make debugging and fixing easier\newline
    -Understanding code is bigger problem than understanding test code\newline
    --> If good tests can be generated, then these may help in understanding the code.\newline
    -Only half of the respondents enjoy writing unit tests\newline
    --> There is a need for tools that rise the enjoyment\newline
    -Maintaining unit tests is found harder than writing unit tests\newline
    --> Need for easier maintaining. Can better readability, Easy parametrization and code repetition removal practices help?\newline\newline

    Automatically Documenting Unit Test Cases:\newline
    -Majority of developers (60.38percent) found understanding of unit test cases to be at least moderately difficult\newline
    -Developers find up-to-date documentation and comments within test cases to be useful\newline
    -Writing comments to unit tests is a practice rarely or never done\newline
    -In order to effectively maintain test cases, it is important that developers understand the impact of each unit
     test case and the particular functionality that it aims to test.\newline
    -Developers feel that tests should be self-documentating\newline
    -More than half of the developers indicated a difficulty of moderate to very hard in terms of understanding unit tests.\newline
    -Emphasizing this importance, we observed that 89.15percent of developers agree or strongly agree that maintaining test cases
      impacts the quality of the system. This suggests that developers could benefit from tools that support them in maintaining unit test
       cases during software evolution and maintenance\newline\newline

   Per Runeson: A Survey of Unit Testing Practices\newline
   -Unit tests are documented in test code rather than in text\newline
   -Unit test motivation in agile: test suites could function as a specification\newline
   -Maintaining was found to take much effort\newline
   -Developer motivation working with unit tests needs improvement\newline
\section{Research hypothesis}
    Discussion about research hypothesis (BD-testing vs JUnit) and how it relates to problems found (8) and benefits mentioned in section 6-7.\newline
    -Developers will write more granular test cases\newline
        %-Ubiquitous language
        %-BDD Helps to shift viewpoint from 1-1 relationship between test-code and only one test method per function
    -Developers will find easier to understand test cases\newline
        %-More comments, longer descriptions naturally in use through ubiquitous language
        %-Test Code should describe the behaviors of the object
    -Developers will find it easier to maintain code\newline
         %-Test Code should be part of system’s documentation
         %-Test Code should have less repetition
    -Developers will perceive working with low level automated testing more enjoyable\newline
