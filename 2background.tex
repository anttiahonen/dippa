\chapter{Background}
\label{chapter:background} 

The problem must have some background, otherwise it is not
interesting.  You can explain the background here. Probably you should
change the title to something that describes more the content of this
chapter. Background consists of information that help other masters of
the same degree program to understand the rest of the thesis.

Transitions mentioned in Section~\ref{section:structure} are used also
in the chapters and sections. For example, next in this chapter we
tell how to use English language, how to find and refer to sources,
and enlight different ways to include graphics in the thesis.

1. Quality assurance in general
2. Automated testing benefits & drawbacks
3. JVM, Java testing with JUnit
3. Behavior driven development
4. Behavior driven low lvel testing frameworks
    -Spec family / Rspec & Spectrum
            -Rspec / BDD Helps to shift viewpoint from 1-1 relationship between test-code and only one test method per function
    -Spock
5. Related research: Analysis of problems found from automated testing surveys
    -There clearly is potential for unit testing research to help developers produce better tests that make debugging and fixing easier
    -Understanding code is bigger problem than understanding test code
    → If good tests can be generated, then these may help in understanding the code.
    -Only half of the respondents enjoy writing unit tests
    → There is a need for tools that rise the enjoyment
    -Maintaining unit tests is found harder than writing unit tests
    → Need for easier maintaining. Can better readability, Easy parametrization and code repetition removal practices help?

    -Majority of developers (60.38%) found understanding of unit test cases to be at least moderately difficult
    -Developers find up-to-date documentation and comments within test cases to be useful
    -In order to effectively maintain test cases, it is important that developers understand the impact of each unit test case and the particular functionality that it aims to test.
    -More than half of the developers indicated a difficulty of “moderate” to “very hard” in terms of understanding unit tests.
     Emphasizing this importance, we observed that 89.15% of developers “agree” or strongly agree” that maintaining test cases
      impacts the quality of the system. This suggests that developers could benefit from tools that support them in maintaining unit test
       cases during software evolution and maintenance

6. Discussion about research hypothesis (BD-testing vs JUnit) and how it relates to problems found (5) and benefits mentioned in section 3-4.
    -Developers will write more granular test cases
        -Ubiquitous language
        -BDD Helps to shift viewpoint from 1-1 relationship between test-code and only one test method per function
    -Developers will find easier to understand test cases
        -More comments, longer descriptions naturally in use through ubiquitous language
        -Test Code should describe the behaviors of the object
    -Developers will find it easier to maintain test cases
         -Test Code should be part of system’s documentation
         -Test Code should have less repetition
    -Developers will perceive working with low level automated testing more enjoyable
